\chapter{Introduction}
This document a long with the QDSP user manual, is the result of a self study project in domain-specific languages (DSL). This activity will focus on creating a DSL that will allow for the creation of functioning machine code for the QDSP which has been developed at MMMI SDU. This DSP has been created to allow easier transfer of regulations algorithms from the computer to the FPGA, with a minimum of changes to the gateware, in order to ease the development of code on the QDSP. The key challenge is thus in the automatic generation of code, not in programming of FPGAs.

In order to accomplish the desired tasks a compiler will need to be crated that is capable of decoding a custom language, and create both a VHDL module containing the QDSP and the code it should run. The compiler should also be able to create the machine code for the DSP allowing it to be programmed onto the DSP at run-time. It would be preferable if the compiler was capable of optimizing the DSP, so only the instructions that are needed gets implemented thereby conserving space on the FPGA, both by not having the logic and allowing the address-busses to operate at a reduced size.