\chapter{Conclusion}
During the project a DSL, with accompany compiler have been designed and implemented. This compiler is capable of creating both a VHDL module containing the QDSP and external programming files. This QDSP can be optimized to only include the instructions required to execute a given program. The language that has been created resembles C thus making it familiar to anyone with experience in C-like programing languages. The instruction-set of the QDSP has been update to allow for jumps in the program code, allowing for a greater flexibility in the user created code. Further more the QDSP has updated to allow run-time programming whereby users can experiment with the code on the DSP without having to rebuild the entire VHDL project.

During the development of the DSL several features have been rejected due to the scope of implementing them, among these are the ability for the run-time programming to be persistent so the program does not revert when restarting the FPGA; division using a IP core, allowing for arbitrary divisors instead of only power of 2 divisors; and creation of variables within loops of if sentences, that gets freed when the section ends.