% Add packages
\usepackage[T1]{fontenc}
\usepackage[utf8x]{inputenc}
\usepackage[english]{babel}
\usepackage{graphicx}
\usepackage{microtype}
\usepackage{lmodern}
\usepackage{lipsum}
\usepackage{makeidx} % Indexing package
\usepackage[intoc]{nomencl} % Nomenclature package
\usepackage{color}
\usepackage[table]{xcolor}
\usepackage{xcolor,calc}
\usepackage{mathpazo}
\usepackage[crop=pdfcrop]{pstool}
\usepackage{multirow}
\usepackage{datatool}
\usepackage{textcomp}
\usepackage{rotating}
%\usepackage[version=3]{mhchem}
\usepackage{relsize}

\usepackage{textpos}
\usepackage{indentfirst}
\usepackage{listings}
\usepackage{slashbox} %grants acces to slash box for tables
\usepackage{float} % Allow the use of H in floats (other options: http://texblog.net/latex-archive/graphics/prevent-floating-image-figure-table/)
\usepackage{pgf}
\usepackage{tikz}
\usepackage{tikz-uml}
%	\usetikzlibrary{arrows}
%	\usetikzlibrary{automata}
\usepackage{footnote}
\usepackage{enumitem}
\usetikzlibrary{  arrows,%
  automata,
  shapes.misc,% wg. rounded rectangle
  shapes.arrows,%
  chains,%
  matrix,%
  positioning,% wg. " of "
  scopes,%
  decorations.pathmorphing,% /pgf/decoration/random steps | erste Graphik
  shadows%
}
\usetikzlibrary{positioning}
\usetikzlibrary{shapes,arrows,decorations.markings}
\usetikzlibrary{shapes.geometric}
\usetikzlibrary{calc}
\usepackage{amsmath,bm,times}
\usetikzlibrary{fit}
\makeatletter
\tikzset{
  fitting node/.style={
    inner sep=0pt,
    fill=none,
    draw=none,
    reset transform,
    fit={(\pgf@pathminx,\pgf@pathminy) (\pgf@pathmaxx,\pgf@pathmaxy)}
  },
  reset transform/.code={\pgftransformreset}
}
\makeatother

%\usepackage[lofdepth,lotdepth]{subfig}
\usepackage[margin=1cm]{caption}
%\usepackage{caption}
\usepackage{courier}
\usepackage{todonotes}
\usepackage{amsmath}

%Must be last include package
\usepackage[bookmarksnumbered=true,
						colorlinks=true, 
						linkcolor=black, 
						citecolor=black, 
						urlcolor=blue,
						runcolor=black,
						filecolor=black,
						menucolor=black
]{hyperref} 
						
						
\graphicspath{{pictures/}}

\everymath{\displaystyle}

% Avoid a warning
\pdfminorversion=5

% Define layout dimensions

\setlrmarginsandblock{35mm}{25mm}{*}
\setulmarginsandblock{30mm}{30mm}{*}
\setheadfoot{8mm}{10mm}
\checkandfixthelayout
\OnehalfSpacing

\setaftersecskip{1\onelineskip plus 3ex minus 3ex}
\setaftersubsecskip{1\onelineskip plus 3ex minus 3ex}
\setbeforesecskip{-1\onelineskip plus -3ex minus -3ex}
\setbeforesubsecskip{-1\onelineskip plus -3ex minus -3ex}

\EndPreamble

%\usepackage{natbib} % Bibliography package
%\usepackage{siunitx} % Should stay after \EndPreamble

% Create a theorem environment
%\usepackage{amsthm}
\newtheorem{theorem}{Theorem}



\makechapterstyle{combined}{
  \setlength{\beforechapskip}{0cm}
	\setlength{\midchapskip}{-60pt}
  \setlength{\afterchapskip}{1.25cm}
  \renewcommand*{\printchaptername}{}
  \renewcommand*{\chapnumfont}{\normalfont\sffamily\bfseries\fontsize{65}{0}\selectfont}
  \renewcommand*{\printchapternum}{\flushright\chapnumfont\textcolor[rgb]{.0,.0,.0}{\thechapter}}
  \renewcommand*{\chaptitlefont}{\normalfont\sffamily\HUGE\bfseries}
%  \renewcommand*{\chapnumfont}{\normalfont\sffamily\bfseries\fontsize{80}{0}\selectfont}
%  \renewcommand*{\printchapternum}{\flushright\chapnumfont\textcolor[rgb]{.2745,.3019,.5176}{\thechapter}}
%  \renewcommand*{\chaptitlefont}{\normalfont\sffamily\HUGE\bfseries}
  \renewcommand*{\printchaptertitle}[1]{%
    \raggedright\chaptitlefont\parbox[t]{\textwidth-3cm}{\raggedright##1}}
}

\chapterstyle{combined}

% Enable indexing
\makeindex

% Enable nomenclature
\makenomenclature


% Make a list of appendices
\newlistof{listofappendices}{loa}{List of Appendices}
\newcounter{appCnt}
\renewcommand{\theappCnt}{\Alph{appCnt}}
\newcommand{\appChapter}[1]{%
 \refstepcounter{appCnt}%
 \chapter*{\theappCnt.\ #1}%
 \addcontentsline{loa}{appCnt}{\numberline{\theappCnt}#1}
}
\newlistentry{appCnt}{loa}{0}
\newlistentry[appCnt]{appSec}{loa}{1}
\newcommand{\appSection}[1]{%
 \refstepcounter{appSec}%
 \section*{\theappSec.\ #1}%
 \addcontentsline{loa}{appSec}{\protect\numberline{\theappSec}#1}
}
\setcounter{loadepth}{2}
\newcounter{pseudo}
\renewcommand{\thepseudo}{\Alph{pseudo}}


% Enable line numbering
%\usepackage{lineno}
%\pagewiselinenumbers
%\modulolinenumbers[5]

% Change table of contents name
\renewcommand*{\contentsname}{Table of Contents}
\maxtocdepth{subsection} % Subsections and above

% Make floats name bold
\captionnamefont{\bfseries}

% Get a signature command
\makeatletter
\newcommand*{\getlength}[1]{\strip@pt#1}
\makeatother

\newlength{\signlength}
\setlength{\signlength}{0.5\textwidth}

\newcommand{\signature}[1]{%
\noindent \line(1,0){\getlength{\signlength}}\\
\noindent #1
}

% Add measuring units to nomenclature
\newcommand{\nomunit}[1]{\renewcommand{\nomentryend}{\hspace*{\fill}#1}}

\captionstyle{\OnehalfSpacing}

\setlength{\belowcaptionskip}{10pt plus 5pt minus 5pt}

% Company logo
\def\bCompanyLogo{false}

% A fix for memoir-kluwer
\renewcommand{\bf}{\textbf}

% Subfigures
\newsubfloat{figure}

% Set listings
%\lstset{numbers=left, numberstyle=\tiny, stepnumber=2, numbersep=5pt, breaklines=true, basicstyle=\small, language=C}
\definecolor{light-gray}{gray}{0.95}
%\lstset{
%language=VHDL,                  % choose the language of the code
%%extendedchars=true,
%frame=shadowbox,	
%basicstyle=\small,       % the size of the fonts that are used for the code
%numbers=left,                   % where to put the line-numbers
%numberstyle=\tiny,      % the size of the fonts that are used for the line-numbers
%stepnumber=1,                   % the step between two line-numbers. If it's 1
%                                % will be numbered
%numbersep=5pt,                  % how far the line-numbers are from the code
%backgroundcolor=\color{light-gray},    % choose the background color. Youmust
%showtabs=false,       	       % show tabs within strings adding
%tabsize=4,	       		       % sets default tabsize to 2 spaces
%breaklines=true,               % sets automatic line breaking
%breakatwhitespace=false,       % sets if automatic breaks should only happen at
%captionpos = b,
%}
\renewcommand{\lstlistingname}{Code Sample}
\lstset{
        basicstyle=\footnotesize\ttfamily, % Standardschrift
        numbers=left,               % Ort der Zeilennummern
        numberstyle=\color{blue!20!black!30!green}\tiny\ttfamily,          % Stil der Zeilennummern
        %stepnumber=2,               % Abstand zwischen den Zeilennummern
        numbersep=5pt,              % Abstand der Nummern zum Text
        tabsize=2,                  % Groesse von Tabs
        extendedchars=true,         %
        breaklines=true,            % Zeilen werden Umgebrochen
        keywordstyle=\color{red},
%   		frame=b,         
%        keywordstyle=[1]\textbf,    % Stil der Keywords
%        keywordstyle=[2]\textbf,    %
%        keywordstyle=[3]\textbf,    %
%        keywordstyle=[4]\textbf,   \sqrt{\sqrt{}} %
        stringstyle=\color{blue}\ttfamily, % Farbe der String
        showspaces=false,           % Leerzeichen anzeigen ?
        showtabs=false,             % Tabs anzeigen ?
        xleftmargin=17pt,
        framexleftmargin=17pt,
        framexrightmargin=5pt,
        framexbottommargin=4pt,
        %backgroundcolor=\color{lightgray},
        showstringspaces=false      % Leerzeichen in Strings anzeigen ?     
}
%
\lstloadlanguages{% Check Dokumentation for further languages ...
%        %[Visual]Basic
%        %Pascal
%        C,
%        %C++
%        %XML
%        %HTML
%        Java
}

\lstdefinelanguage{nxc}[]{c}
{morekeywords={bool,string, mutex, Wait},
sensitive=false,
deletecomment=[s]{/*}{*/},
%morestring=[b]{/*}{*/}
}

\lstdefinelanguage{UCF}[]{VHDL}
{morekeywords={LOC,NET, PULLUP},
sensitive=false,
morecomment=[l]{\#}
%morestring=[b]{/*}{*/}
}

\lstdefinelanguage{qdsp}[]{c}
{morekeywords={var, const, define, config},
sensitive=false,
deletecomment=[s]{/*}{*/},
%morestring=[b]{/*}{*/}
}

\lstdefinestyle{c}{
					language=c,
					basicstyle=\scriptsize \ttfamily \color[HTML]{000000},
					keywordstyle=\scriptsize \ttfamily \color[HTML]{8000FF},
					identifierstyle=\scriptsize \ttfamily \color[HTML]{000000}, 
					commentstyle=\scriptsize \color[HTML]{008000},
					stringstyle=\scriptsize \ttfamily \color[HTML]{808080},
					backgroundcolor=\color[rgb]{0.97,0.97,0.97},
					breaklines=true,%			
					rangeprefix={\/*\ \{\ },% curly left brace plus space
					rangesuffix={\ \}\ *\/},% space plus curly right brace
					rangebeginsuffix={\ start},
					rangeendsuffix={\ end},
					includerangemarker=false
					}
					
\lstdefinestyle{csharp}{
					language=[Sharp]C,
					belowcaptionskip=-1pt,
					basicstyle=\scriptsize \ttfamily \color[HTML]{000000},
					keywordstyle=\scriptsize \ttfamily \color[HTML]{8000FF},
					identifierstyle=\scriptsize \ttfamily \color[HTML]{000000}, 
					commentstyle=\scriptsize \color[HTML]{008000},
					stringstyle=\scriptsize \ttfamily \color[HTML]{808080},
					backgroundcolor=\color[rgb]{0.97,0.97,0.97},
					breaklines=true,%			
					rangeprefix={\/*\ \{\ },% curly left brace plus space
					rangesuffix={\ \}\ *\/},% space plus curly right brace
					rangebeginsuffix={\ start},
					rangeendsuffix={\ end},
					includerangemarker=false
					}
					
\lstdefinestyle{vhdl}{
					language=VHDL,
					belowcaptionskip=-1pt,
					basicstyle=\scriptsize \ttfamily \color[HTML]{000000},
					keywordstyle=\scriptsize \ttfamily \color[HTML]{0000FF},
					identifierstyle=\scriptsize \ttfamily \color[HTML]{000000}, 
					commentstyle=\scriptsize \color[HTML]{008000},
					stringstyle=\scriptsize \ttfamily \color[HTML]{808080},
					backgroundcolor=\color[rgb]{0.97,0.97,0.97},
					breaklines=true,%			
					rangeprefix={--\ \{\ },% curly left brace plus space
					rangesuffix={\ \}},% space plus curly right brace
					rangebeginsuffix={\ start},
					rangeendsuffix={\ end},
					includerangemarker=false
					}
					
\lstdefinestyle{ucf}{
					language=UCF,
					belowcaptionskip=-1pt,
					basicstyle=\scriptsize \ttfamily \color[HTML]{000000},
					keywordstyle=\scriptsize \ttfamily \color[HTML]{0000FF},
					identifierstyle=\scriptsize \ttfamily \color[HTML]{000000}, 
					commentstyle=\scriptsize \color[HTML]{008000},
					stringstyle=\scriptsize \ttfamily \color[HTML]{808080},
					backgroundcolor=\color[rgb]{0.97,0.97,0.97},
					breaklines=true,%			
					rangeprefix={\#\ \{\ },% curly left brace plus space
					rangesuffix={\ \}},% space plus curly right brace
					rangebeginsuffix={\ start},
					rangeendsuffix={\ end},
					includerangemarker=false
					}

\DeclareCaptionFont{white}{\color{white}}
\DeclareCaptionFormat{listing}{\colorbox[cmyk]{0.43, 0.35, 0.35,0.01}{\parbox{\textwidth}{\hspace{15pt}#1#2#3}}}
\captionsetup[lstlisting]{format=listing,labelfont=white,textfont=white, singlelinecheck=false, margin=0pt, font={bf,footnotesize}}
%



\graphicspath{{./pictures/}}
\setsecnumdepth{subsection}


\setsecheadstyle{\normalfont\sffamily\huge\bfseries}
\setsubsecheadstyle{\normalfont\sffamily\LARGE\bfseries}
\setsubsubsecheadstyle{\normalfont\sffamily\Large\bfseries}

\renewcommand{\bibsection}{\chapter{Bibliography}}

\definecolor{table-gray}{rgb}{0.9,0.9,0.9}


%\hypersetup{
%    colorlinks,
%    citecolor=black,
%    filecolor=black,
%    linkcolor=black,
%    urlcolor=black
%}


%-----------------------
% Functions by Jon
%-----------------------

\newcommand*{\myref}[2]{{\hyperref[#2]{#1 \ref*{#2}}}}
\newcommand*{\figref}[1]{\myref{figure}{#1}}
\newcommand*{\tabref}[1]{\myref{table}{#1}}
\newcommand*{\coderef}[1]{\myref{code sample}{#1}}
\newcommand*{\secref}[1]{\myref{section}{#1}}

\newcommand*{\mono}[1]{{\ttfamily #1}}

\newcommand*{\tosNetTabular}[1]{
\DTLsetseparator{;}
\DTLloaddb[keys={register,bregister,word0,word1,word2,word3}]{#1}{#1}
\begin{tabular}{|c|*{5}{c|}}\hline
	\multicolumn{2}{|c|}{
    \multirow{2}{*}{\backslashbox{Register}{Word}}} &0&1&2&3 \\ \cline{3-6}
    \multicolumn{2}{|c|}{} &0b00&0b01&0b10&0b11 \\\hline
    \DTLforeach{#1}{
    	\register=register,\bregister=bregister,\worda=word0,\wordb=word1,\wordc=word2,\wordd=word3}{%
	  	\DTLiffirstrow{}{\\} \register & \bregister & \worda & \wordb & \wordc & \wordd }
   \\\hline
\end{tabular}
}

\newcommand{\makepoints}[5]{
    \foreach \k in {1, ..., #5}
    {
        \pgfmathsetmacro{\dist}{\k * (1 / (#5 + 1))}
        \coordinate (#1 #4 \k) at ($(#1.#2)!\dist!(#1.#3)$);
    }
}

\newcommand{\makepointers}[6]{
    \foreach \k in {1, ..., #6}
    {
        \pgfmathsetmacro{\dist}{\k * (1 / (#6 + 1))}
        \coordinate (#1 #2 #5 \k) at ($(#3)!\dist!(#4)$);
    }
}
\newcommand{\makesqpointers}[6]{
	\makepointers{#1}{#2}{#1.west  |- #2.north}{#1.west  |- #2.south} {left}   {#3};
	\makepointers{#1}{#2}{#1.north -| #2.west} {#1.north -| #2.east}  {top}    {#4};
	\makepointers{#1}{#2}{#1.east  |- #2.north}{#1.east  |- #2.south} {right}  {#5};
	\makepointers{#1}{#2}{#1.south -| #2.west} {#1.south -| #2.east}  {bottom} {#6};
}
\newcommand{\makesqpoints}[5]{
	\makepoints{#1}{north west}{south west}{left}	{#2};
	\makepoints{#1}{north west}{north east}{top}		{#3};
	\makepoints{#1}{north east}{south east}{right}	{#4};
	\makepoints{#1}{south west}{south east}{bottom}	{#5};
}
